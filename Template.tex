%% "LaTeX de Plutón" (c) by Ignacio Slater M.
%% "LaTeX de Plutón" is licensed under a
%% Creative Commons Attribution 4.0 International License.
%% You should have received a copy of the license along with this
%% work. If not, see <https://creativecommons.org/licenses/by/4.0/>.
\documentclass{article}
  \usepackage{import} % This will make our life easier
  \import{preamble}{Packages}
  \import{preamble}{Definitions}
  \import{preamble}{config}

  \setup { 
    title = {\textit{Unifiktion}: Puntuaciones de usuarios para \enquote{obras creativas} en la web 
      semántica.},
    subtitle = { Propuesta de Investigación },
    author = {
      \textbf{Ignacio Slater} \\
      \textit{Departamento de Ciencias de la Computación} \\
      Universidad de Chile \\
      \url{ignacio.slater@ug.uchile.cl}
    },
    logo = {logos/LogoDCC.pdf},
    location = {Santiago, Chile},
    date = \today,
    % Please, I ask you for all that's precious in the world, always include a version for your 
    % document.
    version = 0,
    build = 1,
    commit = 1
  }

\begin{document}
  \begin{titlepage}
    \centering
    % Some vertical space before the title (note the *, it's needed to add space at the beginning 
    % of the page).
    \vspace*{2cm}
    \titleblock [2cm]
    \inputlogo [3.5cm]
    \vspace{1cm}  % Some vertical space after the logo.
    \authorblock
    \vfill  % Fills the page so the location and date are at the bottom.
    \location \\
    \dateblock \\
    \footnotesize { \texttt{\fullversion} }
  \end{titlepage}
  \pagenumbering{Roman}
  \attributionpage

  \import{contents}{Abstract.tex}
  \newpage
  \tableofcontents

  \newpage
  \clearpage
  \pagenumbering{arabic}
  
  \section{Introducción}
  Una «obra creativa» (este es el término que se usa en Wikidata) es toda creación artística «única», como pueden ser
libros, películas, música, etc. Actualmente existen muchas bases de datos de obras creativas. Wikidata logra muy
bien agrupar este tipo de trabajos, pero se centra en información «objetiva». Por otro lado, hay muchísimos sitios
que sí entregan un servicio de este tipo y que generalmente incluyen datos como los scores dados por los usuarios a
cada obra (IMDB, MAL, Rawg, Goodreads, etc).
Una de las funcionalidades que ofrecen las bases de datos que integran datos de usuarios es la de hacer recomendaciones en base a estos datos («Los usuarios que disfrutaron de X también disfrutaron de Y»). Acá podemos apreciar
una limitación ya que al centrarse en un solo tipo de contenidos, solo se pueden hacer recomendaciones dentro de esa
misma categoría. Otra «desventaja» que viene de esta restricción es que no se puede representar correctamente el
concepto de «franquicia», ya que estas pueden estar formadas por distintos tipos de contenido; por ejemplo, Naruto
es una franquicia formada por mangas, anime, novelas y videojuegos (y música si contamos las bandas sonoras
como parte de la franquicia). Wikidata, por otro lado, define el concepto de franquicia y la relación entre obras y
la franquicia a la que pertenecen.
El problema a solucionar es que no existe una forma estándar que permita relacionar los datos «subjetivos» de las
bases de datos especializadas con los datos «objetivos» de Wikidata, limitando enormemente la usabilidad de dichos
datos.
Para enfrentar este problema, se propone extender la ontología de Wikidata para incorporar datos de usuarios
de dos de las bases de datos anteriormente mencionadas,1
tomando una muestra de 50 obras relevantes por sitio.
Finalmente, se propone validar la ontología propuesta verificando que se pueda usar para calcular métricas (promedio
y desviación estándar) a franquicias.
  \section{Estado del arte}
  \section{Objetivos e hipótesis}
  \section{Metodología}
  \section{Resultados esperados}

  \printbibliography
\end{document}
